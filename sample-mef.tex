\documentclass{maff}
\makeatletter
\makeatother
\def\headtext{Recent Macro-economic Indicators (GDP Growth Rate)}
\begin{document}
\frontmatter
\tableofcontents
\listoffigures
\listoftables
\mainmatter
\pagenumbering{khmer}
\chapter{សូចនាករម៉ាក្រូសេដ្ឋកិច្ច}
\section{ការវាយតម្លៃ​កំណើនសេដ្ឋកិច្ច​កម្ពុជា​ក្នុង​ឆ្នាំ ២០១៧}
\begin{figure}[H]
	\centering
%	\fcolorbox{blue}{white}{\includegraphics[width=\linewidth]{mef}}
	\caption{របាយការណ៍ឆ្នាំ ២០១៧}
\end{figure}
ទោះបី​សេដ្ឋកិច្ចពិភពលោក​កំពុងស្ថិត​ក្នុង​ភាពមិនប្រាកដប្រជា​ជាច្រើន ដែល​រួមមាន​ទាំង​ហានិភ័យ និង​បញ្ហា​ប្រឈម ទាំង​ក្នុងប្រទេស​អភិវឌ្ឍន៍ និង​ប្រទេស​កំពុងអភិវឌ្ឍន៍ សេដ្ឋកិច្ចកម្ពុជា​នៅតែ​អាច​បន្ត​កំណើន​ខ្ពស់ និង​រឹងមាំ ក្នុង​រយៈពេល​ខ្លី និង​មធ្យម​ខាងមុខនេះ។ ជាមួយនឹង​សមិទ្ធផល​នេះ ធនាគារ​អភិវឌ្ឍន៍អាស៊ី​បានចាត់ទុក​ប្រទេស​កម្ពុជា ជា​ប្រទេសមួយ​ក្នុងចំណោម​ប្រទេស​ដែលមាន​កំណើនសេដ្ឋកិច្ច​លឿន​បំផុត​នៅ​តំបន់​អាស៊ី ហើយ​បាន​ប្រសិទ្ធ​នាម​ប្រទេស កម្ពុជា ជា ``ខ្លា​សេដ្ឋកិច្ច​ថ្មី​នៅ​អាស៊ី'' ទៀតផង។ ក្នុង​ឆ្នាំ ២០១៦ កន្លងទៅ កម្ពុជា​រក្សាបាន​នូវ​កំណើនសេដ្ឋកិច្ច​ក្នុង​អត្រា ៧,០\% ដោយសារ​ការ​ងើបឡើង​វិញ​នៃ​កំណើន​ក្នុង​វិស័យ​កសិកម្ម បន្ទាប់ពី​ស្ទើរតែ​គ្មាន​កំណើន​សោះ​ក្នុង​រយៈពេល​បី​ឆ្នាំ​ចុងក្រោយ​នេះ ខណៈដែល​កំណើន វិស័យ​ឧស្សាហកម្ម និង​សេវាកម្ម​មាន​ភាព​ថមថយ​បន្តិច។ ការ​ថមថយ​នៃ​កំណើន​ក្នុង​វិស័យ​ឧស្សាហកម្ម និង​សេវាកម្ម​នេះ បានបង្ហាញ​ពី​ភាពចាំបាច់ និង​បន្ទាន់​នៃ​ការអនុវត្ត​វិធានការ​គោលនយោបាយ​កែទម្រង់​រចនាសម្ព័ន្ធ​សេដ្ឋកិច្ច និង​ការធ្វើ​ពិពិធកម្ម ក្នុង​គោលដៅ​ស្វែងរក​ជន្ទល់​ថ្មីៗ ដែលមាន​តម្លៃបន្ថែម​ខ្ពស់​ដល់​សេដ្ឋកិច្ច ដើម្បី​បន្ត​ធានា​ចីរភាព​នៃ​កំណើន​ក្នុង​រយៈពេល​មធ្យម និង​វែង។ យោងតាម​ទិន្នន័យ​ស្ថិតិ​ផ្លូវការ​ដើមឆ្នាំ ២០១៧ របស់​ក្រសួង-ស្ថាប័ន​ពាក់ព័ន្ធ, សេដ្ឋកិច្ចកម្ពុជា ក្នុង​ឆ្នាំ ២០១៧ ត្រូវបាន​ប៉ាន់ស្មាន​ថា នឹង​រក្សាបាន​កំណើន​ក្នុង​រង្វង់ ៧,០\%។ ក្នុងនោះ វិស័យ​ឧស្សាហកម្ម​នៅតែមាន​កំណើន​រឹងមាំ ក្នុង​អត្រា​ពីរ​ខ្ទង់​ដដែល ពោលគឺ​ប្រមាណ ១០,៧\% ដែល​នឹង​បន្ត​ដើរតួនាទី​ជា​ក្បាលម៉ាស៊ីន​ជំរុញ​កំណើនសេដ្ឋកិច្ច​ទីមួយ ដឹកមុខ​ដោយ​វិស័យ​កាត់ដេរ និង​ការរីក​ធំធាត់​នៃ​វិស័យ​មិនមែន​កាត់ដេរ បើទោះបីជា​វិស័យ​សំណង់​មានការ​ថមថយ​បន្តិច​ក្តី។ វិស័យ​កសិកម្ម​អាច​នឹងមាន​កំណើន​ល្អ ពោលគឺ​ក្នុង​អត្រា ២,០\% បើ​ប្រៀបធៀប​នឹង​ឆ្នាំ ២០១៦ ក្នុង​អត្រា ១,៨\% ដោយសារ​កត្តា​អាកាសធាតុ​ដែលមាន​ស្ថានភាព​ល្អប្រសើរ​ជាង​ឆ្នាំមុន និង​ការ​ងើបឡើង​វិញ​នៃ​តម្លៃ​ទំនិញ ខណៈដែល​វិស័យ​សេវាកម្ម​អាច​នឹងមាន​កំណើន​យឺត​ជាង​ឆ្នាំមុន​ជាមួយនឹង​អត្រាកំណើន ៦,៥\% ដោយសារ​ការបន្ត​ថមថយ​នៃ​វិស័យ​ទេស\\
ចរណ៍ និង​អចលនទ្រព្យ។
\par
ជាមួយនឹង​កំណើន​រឹងមាំ​ក្នុង​ឆ្នាំ ២០១៧ នេះ ផ.ស.ស ថ្លៃ​បច្ចុប្បន្ន​ត្រូវបាន​ប៉ាន់ស្មាន​ថា នឹង​កើនឡើង​ដល់ ២២,២ ប៊ីលាន​ដុល្លារ​អាមេរិក  ដែល​ធ្វើឱ្យ ផ.ស.ស សម្រាប់​មនុស្ស​ម្នាក់​អាច​នឹង​កើន​ឡើងដល់ ១.៤៣៤ ដុល្លារ អាមេរិក។ លើសពីនេះ កម្ពុជា​អាច​រក្សាបាន​នូវ​អតិផរណា ដែល​ស្ថិតក្នុង​កម្រិតទាប និង​អាច​គ្រប់គ្រង​បាន ដោយ អតិផរណា​ជាមធ្យម​ប្រចាំឆ្នាំ ត្រូវបាន​ប៉ាន់ស្មាន​ថា នឹងមាន​អត្រា ៣,៨\% ក្នុង​ឆ្នាំ ២០១៧ កើន​ពី ៣,០\% ក្នុង​ឆ្នាំ ២០១៦ ដោយសារ​និន្នាការ​កើនឡើង​នៃ​ថ្លៃ​ប្រេង​ក្នុង​ទីផ្សារ​អន្តរជាតិ និង​តម្លៃ​ម្ហូបអាហារ​ក្នុងស្រុក។ ជាមួយគ្នានេះដែរ អត្រា​ប្តូរប្រាក់​នៅ​បន្ត​មាន​ស្ថិរភាព​ជាមួយនឹង​ការរំពឹងទុក​ថា ប្រាក់រៀល​នឹង​នៅ​បន្ត​រក្សាបាន​នូវ​តម្លៃ​ប្រមាណ ៤.០៣៧ រៀល​ក្នុង ១ ដុល្លារ​អាមេរិក នៅក្នុង​ឆ្នាំ ២០១៧ នេះ។
\begin{tcolorbox}[title={សម្គាល់!}]
	\strut
	\begin{enumerate}[k]
		\item អត្ថបទនេះដកស្រង់ចេញពីគេហទំព័ររបស់ ក្រសួងសេដ្ធកិច្ច និង ហិរញ្ញវត្ថុ។\\
		\url{http://www.mef.gov.kh}\\
		\item ចំណែកឯការរចនាទំព័រ គឺយកតាមលំនាំសៀវភៅ ``ការចិញ្ចឹមមាន់ស្រុក'' ចេញផ្សាយដោយ ក្រសួងកសិកម្ម រុក្ខាប្រមាញ់ និង នេសាទ។\\
		\url{http://www.maff.gov.kh/agri-tech/47-%E1%9E%80%E1%9E%B6%E1%9E%9A%E1%9E%85%E1%9E%89%E1%9F%92%E1%9E%85%E1%9E%B9%E1%9E%98%E1%9E%9F%E1%9E%8F%E1%9F%92%E1%9E%9C/%E1%9E%80%E1%9E%B6%E1%9E%9A%E1%9E%85%E1%9E%89%E1%9F%92%E1%9E%85%E1%9E%B9%E1%9E%98%E1%9E%98%E1%9E%B6%E1%9E%93%E1%9F%8B.html}
	\end{enumerate}
\end{tcolorbox}
\appendix
\chapter{ផ្សេងៗ}
\backmatter
\begin{thebibliography}{2}
	\bibitem{mef} \url{http://www.mef.gov.kh}
	\bibitem{maff} \url{http://www.maff.gov.kh}
	\bibitem{khtug} \url{https://khtug.blogspot.com}
\end{thebibliography}
\end{document}